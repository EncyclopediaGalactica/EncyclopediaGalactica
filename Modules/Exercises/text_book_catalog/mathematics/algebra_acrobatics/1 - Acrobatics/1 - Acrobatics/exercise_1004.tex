\documentclass{article}
\usepackage{tabularx}
\usepackage{booktabs}
\usepackage{mathtools}
\begin{document}

% exercises

% exercise_type = "skills"
% topic_reference = "mathematics"
% book_reference = "algebra_acrobatics"
% chapter_reference = "acrobatics"
% section_reference = "acrobatics"
% manual_id = 1004

% paragraph title is the question number
\section{Exercise 1004}

% question start
Solve the following equation:
\begin{align*}
	\left( x^2 + 6x + 1\right)\left( x^2 + 6x - 3 \right) = 5
\end{align*}
% question end

% solution start
\begin{align*}
	\left( x^2 + 6x + 1\right)\left( x^2 + 6x - 3 \right) & = 5                                                                \\
	\text{ let } u                                        & = x^2 + 6x                                                         \\
	\left( u + 1 \right)\left(u - 3\right)                & = 5 \text{// substitute u}                                         \\
	u^2 - 2u - 3                                          & = 5 \text{// factor}                                               \\
	u^2 - 2u - 8                                          & = 0                                                                \\
	\left( u - 4 \right)\left( u + 2 \right)              & = 0                                                                \\
	\left( u - 4 \right) = 0                              & \text{ and } \left( u + 2 \right) = 0 \text{// the two solutions } \\
	u = 4                                                 & \text{, } u = -2                                                   \\
	\text{ \textbf{solution 1 for substitution:} } u      & = 4                                                                \\
	x^2 + 6x + 9                                          & = 4 + 9                                                            \\
	\left( x + 3 \right)^2                                & = \sqrt{13}                                                        \\
	x                                                     & = -3 \pm \sqrt{13} \text{ // \textbf{real solution 1}}             \\
	\text{ \textbf{solution 2 for substitution: } } u     & = -2                                                               \\
	x^2 + 6x + 9                                          & = -2 + 9                                                           \\
	\left( x + 3 \right)^2                                & = \sqrt{7}                                                         \\
	x                                                     & = -3 \pm \sqrt{7} \text{ // \textbf{real solution 2}}              \\
\end{align*}

\subparagraph{Notes}
\begin{itemize}
	\item we haven't multiplied the two together because it would result in a complex mess where
	      easy to make mistakes
	\item we used substitution
	\item to find the correct numbers to transform the $$ u^2 - 2u - 8 = 0 $$ to $$ \left( u - 4 \right)\left( u + 2 \right) = 0 $$ we used the trick described below.
	\item the magic to find the integer added to the $$ x^2 + 6x $$ to get a perfect square is
	      described below too.
	\item we haven't used the quadratic formula due to \dots
\end{itemize}
% solution end

% exercises end

\end{document}
