%ankicard
%ankitags mathematics, fundamentals, modeling_with_equations
\subsection{What are the details of modeling with equations?}
%ankifront
\begin{small}
    \begin{tabularx}{1\textwidth}{
            p{\dimexpr1\textwidth\relax}
        }
        \toprule
        What are the details of modeling with equations?
        \\
        \bottomrule
    \end{tabularx}
\end{small}
%ankifront end
%ankiback
\begin{small}
    \begin{tabularx}{1\textwidth}{
            p{\dimexpr1\textwidth\relax}
        }
        \toprule
        \textbf{Identify the variable.}
        Identify the quantity that the problmen asks yo to find. This quantity
        can usually be determined by a careful reading of the question that is
        posed at the end of the problem. Then \textbf{introduce notation} for
        the variable (carr it $x$ or some other letter).
        \\
        \midrule
        \textbf{Translate from Words to Algebra.}
        Read each sentenxe in the problem again, and express all the quantities
        mentioned in the problem in terms of the variable you defined int Step
        1. To organize this information, it is sometimes helpful to \textbf{draw
           a diagram or make a table}.
        \\
        \midrule
        \textbf{Set up the model}.
        Find the crucial fact in the problem that gives a relationship between
        the expressions you listed in Step 2. \textbf{Set up an equation} or a
        model that expresses this relationship.
        \\
        \midrule
        \textbf{Solve the equation and check your answer.}
        Solve the equation, check you answer, and \textbf{state your answer} as
        a sentence.
        \\
        \bottomrule
    \end{tabularx}
\end{small}
%ankiback end
