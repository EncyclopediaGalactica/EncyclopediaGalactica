%ankicard
%ankitags mathematics, fundamentals, modeling_with_equations
\subsection{
how to solve nonlinear inequalities?}
%ankifront
\begin{small}
    \begin{tabularx}{1\textwidth}{
            p{\dimexpr1\textwidth\relax}
        }
        \toprule
        how to solve nonlinear inequalities?
        \\
        \bottomrule
    \end{tabularx}
\end{small}
%ankifront end
%ankiback
\begin{small}
    \begin{tabularx}{1\textwidth}{
            p{\dimexpr1\textwidth\relax}
        }
        \toprule
        Signs of product or quotient: \\
        if a product or a quotient has an \textbf{even} number of
        \textbf{negative} factors, then its valie is \textbf{positive}. \\
        if a product or a quotient has an \textbf{odd} number of
        \textbf{negative} factors, then its value is \textbf{negative}.
        \\
        \midrule
        \textbf{Guidelines for solving nonlinear inequalities}\\
        \textbf{Moce all terms to one side}\\
        If necessary, rewrite the inequality so that all nonzero terms appear on
        one side of the inequality sign. If the nonzero side of the inequality
        involves quotients, bring them to a common denominator.\\
        \textbf{Factor} Factor the nonzero side of the inequality. \\
        \textbf{Find the intervals}\\
        Determine the values for qhich each faxtor is zero. These numbers divide
        the real line into intervals. Llist the intervals that are determined by
        these numbers.\\
        \textbf{Make a table or a diagram} Use \textbf{test values} to make a
        table or diagram of the signs of each factor on each interval. Inteh
        last row of the table determine the sign of the product (or quotient) of
        these factors.\\
        \textbf{Solve} Use the sign table to find the intervals on which the
        inequality is satisfied. Check whether the endpoints of these intervals
        satisfy the inequality. (This may happen if the inequaltiy involves
        $\geq \text{ or } \leq $)
        \\
        \midrule
        \textbf{Important} The factoring technique that is described in the
        guidelines works only if all non-zero terms appear on one side of the
        inequality symbol.
        \\
        \bottomrule
    \end{tabularx}
\end{small}
%ankiback end
