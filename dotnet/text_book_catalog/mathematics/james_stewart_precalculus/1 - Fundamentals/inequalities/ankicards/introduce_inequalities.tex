%ankicard
%ankitags mathematics, fundamentals, modeling_with_equations
\subsection{
Introduce inequalities}
%ankifront
\begin{small}
    \begin{tabularx}{1\textwidth}{
            p{\dimexpr1\textwidth\relax}
        }
        \toprule
        Introduce inequalities
        \\
        \bottomrule
    \end{tabularx}
\end{small}
%ankifront end
%ankiback
\begin{small}
    \begin{tabularx}{1\textwidth}{
            p{\dimexpr1\textwidth\relax}
        }
        \toprule
        Some problems lead to inequalities instead of equalities. An inequality
        looks like an equality, but it represents that the solution is greater
        (greater or equal) or less (less or equal) than a certain value. Based
        on this we can say that the solution of an inequality is not an exact
        number, but an interval.
        \\
        \bottomrule
    \end{tabularx}
\end{small}
%ankiback end
